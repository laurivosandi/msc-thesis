\documentclass{article}
\usepackage[utf8]{inputenc}

\title{Efficient and Reliable Filesystem Snapshot Distribution}
\author{Lauri Võsandi}
\date{January 2015}

\usepackage{natbib}
\usepackage{graphicx}
\usepackage{url}

\begin{document}

\maketitle

\section{Introduction}

TODO

\section{Background}

The foundation of current work was established while author was setting up
the infrastructure to deploy Ubuntu 12.04 LTS on the PC-s of educational
institutions of Tallinn as part of the ongoing efforts of Tallinn Education
Department to switch from proprietary solutions to open-source software.

Puppet was set up to manage Ubuntu workstations remotely. Local IT-support
took the role of bootstrapping the machines and joining them to remote
management server.


\section{Specification}

Customer A runs hundreds of embedded ARM computers for digital signage.
Software is currently updated by mailing the customer an SD-card with
updated software. The customer would prefer to update software and
media over the air but the software update atomicity has to be guaranteed
in order to avoid non-booting machines.

Customer B is about to deploy thousands of Ubuntu netbooks to be used as
remote workstations around the globe. It vital to unroll security updates
as soon as possible, but at the same time it's necessary to guarantee
software update atomicity as the IT helpdesk is lacking in the remote
locations where the machines are used.

Customer C has around thousand PC-s that need to be converted to Ubuntu,
but the budget is lacking and therefore manual labour has to be minimized.
Glitch-free software update mechanism is crucial part of minimizing manual
labour.

\section{Current approaches}

There are various  currently used to manage Linux based workstations.
In this chapter a background of different methods is outlined.
Subsequently a specification is derived from the needs of customers of
the author. 


\subsection{SaltStack}
TODO

\subsection{Puppet and Foreman}

Puppet is a remote management system which features its own declarative
domain-specific language to describe the state of the configuration
\footnote{http://puppetlabs.com/}. Puppet server also known as Puppetmaster
hosts the configuration while managed machines run puppet agent which polls
the puppetmaster at specified interval, usually 30 minutes. Taken actions
are then reported back to the puppetmaster. Puppet agent and puppetmaster
both are written in Ruby and released the latest versions are released under
liberal Apache 2.0 license. Puppet can be used to manage both Linux and
Windows servers and workstations as well.

Foreman is a complete lifecycle management software for physical and virtual
servers. Foreman incorporates Puppet, a custom web interface and provisioning
tools into single unified application. Even without using provisioning features
Foreman makes one of the most feature-complete web interfaces for Puppet
rivaling the Puppet Dashboard.

\subsection{Chef}

Chef is infrastructure automation tool. Chef is written in Ruby and Erlang. Chef uses domain-specific language written in Ruby.

\subsection{Ansible}

Ansible remote management software uses SSH to connect to the nodes which
means there is no agent running on the managed machine, this however makes
it slightly more complicated to use Ansible no manage machines behind NAT.
Ansible is written in Python and it uses state-driven resource 
written in YAML.

\subsection{Clonezilla}

Clonezilla combines various open-source tools into a single cloning suite.
Clonezilla uses partclone utilities  \footnote{http://partclone.org/} to
identify and transfer only used blocks of various filesystems, most notably
NTFS, ext4 and btrfs. Clonezilla can be used to grow fielsystems on the fly
making it possible to use same prepared image for disks of various size. 

\subsection{Jails, OpenVZ and LXC}

Jails have been available in FreeBSD since version 4.x. Jails use chroot
syscall to substitute root filesystem of a process making it possible to
create a restricted environment which is isolated from the rest of the
operating system
\footnote{https://www.freebsd.org/doc/en/books/handbook/jails.html}.

The main issue with jails is that dependencies of the target application have
to be available in the jail root filesystem. For instance a Python application
which has modules loaded before chroot operation could operate without any
files in the chroot, but shell script which relies on several executables need
to have those utilities available in chroot as well. With copy-on-write and de
duplicating filesystems the problem how ever becomes irrelevant as root
filesystem of the chroots can be duplicated with no significant overhead.


\subsection{Docker and Rocket}

Docker started off as a way to automate container deployment and configuration
using containers and cgroups present in Linux kernel. As Docker started to add
features that CoreOS developers deemed excessive an alternative project Rocket
was founded
\footnote{\url{http://www.theregister.co.uk/2014/12/03/coreos_rocket_deep_dive/}}.

\subsection{CoreOS and Ubuntu Core}

CoreOS \footnote{https://coreos.com/} is a rearchitected Linux distribution
which provides minimalist foundation to run containers.
It uses two-partition scheme to provide atomic updates of the root filesystem.
The operating system runs off a read-only filesystem while the other one
can be patched runtime. Reboot or \emph{kexec} can be used to boot into
the updated system.


Ubuntu Core is an Ubuntu flavour tailored towards Internet of Things and as a
container platform. Ubuntu Core introduced root filesystem transactional
updates to Ubuntu using Snappy
\footnote{http://developer.ubuntu.com/en/snappy/}.
Ubuntu Core is designed to run Docker applications and can be used.
Snappy is also plays important role in Ubuntu Phone ecosystem.
Snappy uses OverlayFS (?) to implement transactional updates 

\subsection{apt-btrfs-snapshot and yum-fs-snapshot}

In Ubuntu package repositories there is available \emph{apt-btrfs-snapshot}
\footnote{https://launchpad.net/apt-btrfs-snapshot} package,
which creates snapshot of the root filesystem before every
\emph{apt-get} operation.

\emph{yum-plugin-fs-snapshot}
\footnote{\url{http://man7.org/linux/man-pages/man1/yum-fs-snapshot.1.html}}
is the corresponding package for Fedora and
Red Hat based distributions.


\subsection{OverlayFS}

OverlayFS is a feature introduced in Linux 3.18 which makes it possible to merge contents of two separate mountpoints on the fly.
OpenWrt uses OverlayFS to implement writable jffs2 layer on top of read-onlySquashFS filesystem \footnote{https://git.kernel.org/cgit/linux/kernel/git/torvalds/linux.git/tree/Documentation/filesystems/overlayfs.txt}.

\subsection{LVM, mdadm and dmraid}

Ext4 has been primary filesystem for Linux based workstations and servers for
a while. It provides filesystem primitives such as files, directories,
permissions and timestamping. In order to add redundancy either software RAID
or logical volume management (LVM) can be used.

Software RAID is implemented in Linux by means of \emph{mdadm}. Software RAID
can be used to build RAID1, RAID0, RAID10/01, RAID5 or RAID6 arrays without
dedicated RAID controller which could also impose a vendor lock-in.

LVM enables pooling of drives, mirroring and snapshotting by adding an
abstraction layer on top of physcal disks. Any filesystem that can be deployed
on physical disk can also be deployed on top of LVM's logical volume. The
kernel takes care of mapping logical addresses to corresponding disk's physical
address. The snapshotting feature of LVM however has been claimed to be buggy
\footnote{http://lwn.net/Articles/522073/}.

\subsection{Btrfs and ZFS}

Btrfs and ZFS both are modern copy on write filesystem for Linux which also
fills in the role of volume manager. Btrfs has been claimed to be unstable but
the situation has improved significantly over the past year or two. 

During snapshot send/receive an optimal parent snapshot is identified and that
is used as basis for the differential snapshot. The btrfs stream contains
filesystem operations that are indended to be replayed on a clone of the
parent subvolume: create file, \emph{mkdir}, \emph{mknod}, \emph{mkfifo},
symlink, \emph{link}, \emph{unlink}, \emph{rename}, \emph{rmdir}, open file,
close file, write to file, set/remove extended attributes, truncate file,
\emph{chmod}, \emph{chown}
\footnote{http://git.kernel.org/cgit/linux/kernel/git/kdave/btrfs-progs.git/tree/cmds-receive.c}.

Debian has supported btrfs since Squeeze and has improved support since then
\footnote{https://wiki.debian.org/Btrfs}.  

Facebook has been testing btrfs in production since the April of 2014
\footnote{\url{https://btrfs.wiki.kernel.org/index.php/Production_Users}}.
Chris Mason, a lead developer of Btrfs joined Facebook in the end of 2013
\footnote{\url{http://article.gmane.org/gmane.comp.file-systems.btrfs/30420}}.
RAID5/6 support and improved data scrubbing for Btrfs was released with Linux
3.19 \footnote{http://lkml.iu.edu/hypermail/linux/kernel/1412.1/03583.html}.

\subsection{OpenWrt}
OpenWrt is a Linux distribution for embedded devices which attempts to provide
writable filesystem and package management \footnote{https://openwrt.org/}.
OpenWrt supported hardware list mainly targets routers, but other devices are
listed as well \footnote{http://wiki.openwrt.org/toh/}. OpenWrt can be used to
extend lifetime of equipment that otherwise would be largely unmaintained by the
manufacturer.

Most high-end consumer grade routeres employ 8MB NAND Flash chip which is
directly connected to the SoC without controller
\footnote{http://wiki.openwrt.org/doc/techref/flash.layout}.
The Flash storage is usually partitioned at least as 3 slices:
bootloader, read-only root filesystem, read-write overlay.

The read-only root filesystem contains SquashFS
\footnote{http://squashfs.sourceforge.net/}
which is highly-efficient compressed read-only filesystem that support variety
of compression The read-write overlay partition is formatted as JFFS2
(journalling flash filesystem).

The first method splits internal storage to two partitions: SquashFS
filesystem which contains read-only firmware and JFFS2 formatted partition
which is laid over the SquashFS filesystem using OverlayFS. This method makes
it possible to do factory reset simply by formatting the JFFS2 partition.

\subsection{OpenStack}

TODO

\section{Analysis}

Ubuntu package management fits most scenarios, but there are some rough edges:
package list corruption
\footnote{http://askubuntu.com/questions/532200/14-04-lts-apt-get-segfault},
faulty scripting in packages
\footnote{\url{https://www.linuxquestions.org/questions/showthread.php?s=e0e2f7689f847a56e8cee94a0cafd6bd&p=5216367#post5216367}},
bandwidth overhead etc.

Release upgrades for example from Ubuntu 12.04 to Ubuntu 14.04
have proven to be especially troublesome due to the fact that system libraries
and files are updated and interrupted release upgrade may leave system
in an unusable state.

Debian community has been working hard to provide differential updates for
the packages, but as of February 2015 the efforts have proven fruitless.
Differential updates are applied for package lists
\footnote{\url{https://www.debian-administration.org/article/439/Avoiding_slow_package_updates_with_package_diffs}},
but binary diffs for packages have not implemented yet.
Fedora community has however successfully deployed differential pacakges
\footnote{http://fedoraproject.org/wiki/Features/Presto},
thus reducing the amount of data needed to be transferred during an pacakge update. For bigger software (eg LibreOffice) the lack of differential updates poses a serious concern, especially for low-bandwidth links.

Puppet, SaltStack, Chef, Ansible and other traditional configuration management
fit best the scenario where each node has slightly different configuration and
it makes sense to keep them separate. However provisioning very similar nodes
with for instance Foreman has obvious overhead - each node has to fetch updated
packages independently from the same APT repositories, same has to be done for
application software.

\section{Prototype}

Using Debian, Ubuntu and Gentoo were evaluated as provisioning utility
operating system. With Debian and Ubuntu the resulting PXE bootable image
would have exceeded 100MB.
As of February of 2015 the CoreOS image suffers similar issue -
\emph{vmlinuz}
\footnote{\url{http://stable.release.core-os.net/amd64-usr/current/coreos_production_pxe.vmlinuz}}
and
\emph{initrd}
\footnote{\url{http://stable.release.core-os.net/amd64-usr/current/coreos_production_pxe_image.cpio.gz}}
files required to boot over PXE are correspondingly 24MB and 117MB.
With Gentoo significant tweaking is required, because Gentoo is
mainly targeted for power users.

Using Python to build pseudo-graphic menu-driven user interface was
evaluated and deemed not necessary for the goal as Python runtime and
dependant libraries add about 10MB to the resulting image.
In addition to that \emph{parted} Python bindings were unavailable
in Buildroot package selection.

Buildroot was eventually used to generate an compact 10MB all-in-one
PXE-bootable image. Utilities \emph{dialog} in conjunction with
\emph{curl}, \emph{jq} and others were used to build the user-interface
and Bash was used to program the user-interface logic.
The security model for the initial deployment phase could be improved
as only method of verification of the source is the certificate
authority chain verified by \emph{curl} during the btrfs snapshot
retrieval.

LXC containers are used to bootstrap the template for provisioning.
With btrfs backing store on top of a btrfs filesystem the container
can be saved in an isolated btrfs subvolume which makes it easy to
snapshot the container.
Within the container \emph{puppet apply} and similar methods can be used
to take advantage of already existing configuration management know-how.
Otherwise traditional manual labour can be employed to set up the template.
During the release phase the LXC container is stopped, pre-release scripts
are executed to clean up package cache and temporary files.
Then a read-only btrfs snapshot is generated from the container root filesystem.
At this point new snapshot becomes available for other nodes.

For running nodes a DBus service was written to poll the snapshot server
for updates and another DBus service was written in Python to notify user
about available updates.


\section{Conclusion}

TODO

\section{Future work}

\bibliographystyle{plain}
\bibliography{references}

\end{document}
